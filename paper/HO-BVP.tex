\documentclass[12pt]{article}
\usepackage{epsfig,vin}
\usepackage{amscd,latexsym,txfonts}
\newcommand{\Xp}{{\mathbf X^\prime}}
\newcommand{\X}{{\mathbf X}}
\pagestyle{empty}
\topmargin=-0.5in
\textwidth=6.25in
\oddsidemargin=-0.25in
\textheight=9.5in
\begin{document}
\centerline{\LARGE \bf Intro to Partial Differential Equations Handout} 
\centerline{\Large \bf Boundary Value Problems}
\vspace{1cm}
\normalsize

[Start Section 4.1] Assume the diffusion equation is to be solved over a spatial interval $(0,l)$

\[ \left\{ \begin{tabular}{rl} $u_t = Du_{xx}$, & $(0,l) \times (0,\8)$ \\ 
 & \\
 $u(x,0) = \phi(x)$, & $u(0,t)=u(l,t)=0$ \\ \end{tabular} \right. \]
 
\bsni Now assume the solution has the structure of separated variables: $u(x,t) = T(t) X(x)$. Then the PDE implies

 \[ T'(t) X(x) = D T(t) X''(x) \rightarrow \frac{T'}{T}(t) = \frac{X''}{X}(x) = \lam \]
 
\bsni since a function of $t$ can only be identically equal with a function of $x$ if they are both constant.

\bsni
\begin{tabular}{|p{\textwidth}|} \hline Note that the $t$ equation yields $T(t) = c e^{\lam t}$ regardless of $\lam$'s value.  

\bsni Let $\lam \in \mathbb{C}$ in the $x$ equation

\[ X(x) = c_1 e^{\sqrt{\frac{\lam}{D}}x} + c_2 e^{-\sqrt{\frac{\lam}{D}}x}\]  

The boundary data at $x=0$ implies $c_1 + c_2 = 0$.  The data at $x=l$ shows that if $e^{\sqrt{\frac{\lam}{D}}l} \neq 0$ then 
the only solution is $c_1 = c_2 = 0$, which also satisfies the $x=0$ equation.  This leaves $u(x,t) = 0$, which is not useful.

\bsni To get $e^{\sqrt{\frac{\lam}{D}}l} = 0$, then $\sqrt{\frac{\lam}{D}}$ must be imaginary, or $\lam < 0$.  Real valued solutions 
yield 

\[ X(x) = c_1 \cos\left(\sqrt{\frac{-\lam}{D}}x\right) + c_2 \sin\left(\sqrt{\frac{-\lam}{D}}x\right)\]  

Now the condition at $x=0$ yields $c_1 = 0$ and the one at $x=l$ yields 
$c_2 \sin\left(\sqrt{\frac{-\lam}{D}}x\right) = 0$, which will be satisfied for any $c_2$ if $\sqrt{\frac{-\lam}{D}} = k \pi$, for $k \in \zb$ or 
$\lam_k = -D\left(\frac{k \pi}{l}\right)^2$. \\ \hline \end{tabular} 

\bigskip So, combining arbitrary constants, solutions are $u(x,t) = c e^{-D\left(\frac{k \pi}{l}\right)^2 t} \sin\left(\frac{k \pi}{l}x \right)$.  
These non trivial solutions to $X'' = \lam X$ are {\it eigenvalues} $\lam_k = -D\left(\frac{k \pi}{l}\right)^2$ 
[positive part in the text] and {\it eigenfunctions} $\sin_k(x) \equiv \sin\left(\frac{k \pi}{l} x\right)$ for $k \in \zb^+$.  
By linearity, for a fixed $N \in \mathbb{N}$, the set $F_N = \left\{\sum^N_{k=1} 
c_k \sin_k(x) \right\}$ contains all solutions to the PDE.  Then $u(x,0) = \sum^N_1 c_k \sin_k(x)$, so
this approach will solve BVPs with initial data $\phi(x) \in F_N$. [End Section 4.1] 

\bigskip It is then natural to ask, 

\begin{enumerate}

\item Are $\sin\left(\frac{k \pi}{l} x\right)$ linearly independent, so that they would be a {\it basis} for the
{\it vector space} $F_N$?

\item How much territory does $F_N$ cover in the set of functions? 

\end{enumerate}

The first question can be answered using the inner product $f \cdot g = (f,g) \equiv \int^l_0 f(x)\overline{g(x)}dx$, and denoting
$\sin_k \equiv \sin\left(\frac{k \pi x}{l}\right)$, one can see that $\sin_{k_1} \cdot \sin_{k_2} = 0$ if $k_1 \neq k_2$ by using the identity

\[ \sin(x)\sin(y) = \ohf \cos(x-y) - \ohf \cos(x+y) \]

\noindent so that $\{\sin_k\}$ are {\it orthogonal}, but not orthonormal unless $l =2$ because $\sin_{k_1} \cdot \sin_{k_2} = \ohf l$.  

\bigskip For the second question, note that $F_N \subset F_{N+1}$ so this will be addressed when $N \to \8$.  The projection 
formula from Calc III holds because of the orthogonality, and so the constants $c_k$ ({\it Fourier sine coefficients}) are 
found from 

\[ c_k  = \frac{\int^l_0 \sin\left(\frac{k \pi x}{l}\right) \phi(x) dx}{\ohf l} = \frac{\sin_k \cdot \phi}{\sin_k \cdot \sin_k} \]
  
\bsni Neumann conditions at the boundary yield $\cos$ series [see Section 4.2] while periodic boundary 
conditions yield the full Fourier series [see Section 5.1] over $[-l,l]$.
 
\begin{enumerate}
\item For $\cos$, there is also orthogonality and the same formula for $c_k$ holds with $\sin$ replaced by $\cos$, when $k > 0$.  

\item For $\cos$, when $k = 0$, there is $1 \cdot 1 = l$ and so $\ds{c_0 = \frac{1}{l} \int^l_0 \phi(x) dx}$ which is the average value 
of $\phi$ over $[0,l]$.  

\item Because the vector $1$ has twice the inner product that $\cos_k$, $\sin_k$ have, the $c_0$ formula has an extra 
$\ohf$, which explains textbook formulas with an `extra one half for $a_0$'.

\item $\rule{0pt}{1ex}$ For full, the double sum may be rewritten as $\sum^N_{-N} c_k e^{\frac{i k \pi x}{l}}$ with 

\[ c_k = \frac{\int^l_{-l} e^{\frac{i k \pi x}{l}} \phi(x) dx}{2l} = \frac{e_k \cdot \phi}{e_k \cdot e_k} \]  

Note the inner product issue disappears here: $e_k \cdot e_k = 2l$ for any $k \in \zb$. [see Section 5.2]

\end{enumerate}

\bsni \ul{Variants}:
For an arbitrary interval $[a,b]$, shift to $[0,b-a]$, compute the series using $l =b-a$, then substitute $x+a$ for $x$ in that series.
For inhomogeneous boundary conditions, see Section 5.6.  It can be shown that the approximation by finite linear combinations of 
$\cos_k$, $\sin_k$ is the least squares approximation. Now consider the questions

\begin{enumerate}

\item Do sums of the form $\sum^N_1 c_k \sin_k(x)$ converge when $N \to \8$?   

\item If the previous answer is yes, then how much territory does $\ds{\overline{\cup^{\8} F_{N}}}$ cover in the set of functions? 

\end{enumerate}

\noindent  This will be addressed in Section 5.4 and 5.5.
\end{document}